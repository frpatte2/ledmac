\documentclass{book}
%Edizione Critica
\usepackage[nofamiliar,noend,noeledsec,noledgroup,series={A,B,C}]{reledmac}
\labelpstarttrue
\lineation{page}
\linenummargin{inner}
\Xarrangement[A,B,C]{paragraph}
\Xnumberonlyfirstinline[C]
\Xnumberonlyfirstintwolines[C]
\Xnonbreakableafternumber[C]
\Xnotefontsize[A,B,C]{\footnotesize}
\Xpstart[A,B,C]

\makeatletter
\renewcommand{\thepstart}%
	{{\hspace{3mm}\makebox[0pt][r]%
		{\bf\@arabic\c@pstart}}\hspace{3mm}%
	}
\makeatother

\usepackage{hyperref}

\begin{document}
\beginnumbering
\numberpstarttrue

\pstart%
Quel ramo del lago di Como che volge a mezzogiorno
\pend

\pstart
e il \edtext{ponte}{\Cfootnote{via}}, che ivi congiunge le due rive, par che renda ancor più sensibile all'occhio questa trasformazione, e segni il punto in cui il lago cessa, e l'Adda rincomincia, per ripigliar poi nome di \edtext{lago}{\Cfootnote{fiume}} dove le rive, allontanandosi di nuovo, lascian l'acqua distendersi e rallentarsi in nuovi golfi e in nuovi seni
\pend

\pstart
tra due catene non interrotte di monti, tutto a seni e a golfi
\pend

\pstart
a seconda dello sporgere e del \edtext{rientrare}{\Bfootnote{kjhkjkjh}} di quelli, vien, quasi a un tratto
\pend

\pstart
a ristringersi, e a prender corso e figura di fiume, tra un promontorio a destra, e un'ampia costiera dall'altra parte
\pend

\numberpstartfalse
\endnumbering
\end{document}
