\documentclass[11pt,twoside]{book}
\usepackage{fontspec}
\usepackage{xunicode}
\setmainfont[]{Palatino}
\usepackage[parapparatus]{reledmac}

\AtEveryPend{\vspace{\parskip}}

\Xarrangement[A]{paragraph}
\Xarrangement[B]{paragraph}
\Xarrangement[C]{paragraph}
\Xarrangement[D]{paragraph}

\newcommand{\choiceline}[2]{\linenum{|#1|#2||#1|#2}}
\sublinenumberstyle{alph}
\let\fullstop\relax

\setlength{\parindent}{0pt}

\setstanzaindents{0,0,0,0}
\setcounter{stanzaindentsrepetition}{0}
\numberstanzatrue

\Xstanza
\Xstanzaseparator{.}

\lineation{page}
\firstlinenum{1}
\linenumincrement{1}
\firstsublinenum{1}
\sublinenumincrement{1}

\begin{document}

\beginnumbering
\skipnumbering
\pstart
\centering{||} O{||} \textit{avighnam astu}{||} O{||}\pend

\pstart\noindent{}\edtext{saṅ}{\Bfootnote{em.; sa LOr}} saṅgrahakāri sira movus{|} liṅnira{|} \pend

\stanza
 \setlength{\leftskip}{15mm}\setlength{\rightskip}{15mm}\itshape\hidenumbering \edtext{śūnyaś}{\choiceline{1}{1}\Afootnote{em.; śūnya A}}  ca \edtext{nirbbāṇādhikaḥ{|} śivāṅgatve}{\linenum{|2|1||2|2|}\Afootnote{em; nirbbhāṇādhika{|} śśivaṅgatve A, B}} nirīkṣyate{|}&
\setlength{\leftskip}{15mm}\setlength{\rightskip}{15mm}\itshape\hidenumbering kutaḥ tadvākyam atulaṁ{|} śrutvā devo \edtext{’vatiṣṭhati}{\Afootnote{em; vatiṣṭha{|} ca A, B}} {||} \normalshape 1.2\&[ ]

\pstart nāhan takvanaknaniṅ hulun ri bhaṭāra{|} hana ya pada śūnya{|} ya sinaṅguh ka\-mo\-kṣan{|} ṅa{|} viśeṣa ya{|} ya śiva ṅaranya{|} \textit{nirīkṣyate}{|} katon pva ya de saṅ \edtext{yogīśvara}{\lemma{\ldots{}gīśvara}\Cfootnote{B om.}}{|} sājñā bhaṭāra{|} an maṅkana kottamaniṅ vuvus saṅ ṛṣi{|} ya ta kaṛṅĕ de bhaṭāra{||} 1.2\pend

\pstart\textit{\edtext{kenopāyena bhagavan}{\choiceline{25}{1}\Afootnote{em.; meyopāyena bhagavān A}\Dfootnote{unmetrical}}}{|}\textit{sukhayogasya lakṣaṇaṁ}{|} \edtext{nihan}{\Bfootnote{conj.; nāhan A}} takvananiṅ hulun ri kita bha\-ṭā\-ra{|} ndya kunaṅ upāyanikā saṅ paṇḍita{|} mataṅyan kapaṅguhāvaknikaṅ sukhādhyātmika{||} 10.25\pend

\endnumbering
\end{document}
