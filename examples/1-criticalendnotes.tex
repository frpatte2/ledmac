\documentclass{article}
\usepackage[osf,p]{libertinus}
\usepackage{microtype}
\usepackage[pdfusetitle,hidelinks]{hyperref}
\usepackage[english, main=latin]{babel}
\babeltags{english = english}

\usepackage[series={A,B},nocritical,noeledsec,nofamiliar,noledgroup]{reledmac}
\Xendparagraph[B]
\begin{document}

\begin{english}
\date{}
\title{Critical endnotes}
\maketitle
\begin{abstract}
This file provides examples of critical endnote usage with reledmac. 
A critical note is associated with a lemma by marking it with \verb+\edtext+ and referenced by the line and page numbers of the lemma.
When a critical note refers to a long lemma, we can use \verb+\lemma+ to produce an abbreviated form.

Here we use two series of critical notes. 
\begin{itemize}
\item Each note of series A has its own paragraph. 
\item The notes of series B are arranged in the same paragraph.
\end{itemize}
\end{abstract}
\end{english}

\beginnumbering
\pstart
\edtext{Lorem}{
  \Aendnote{A critical note}
  \Aendnote{Critical note in series A}
  \Aendnote{Critical note in series A}
  \Bendnote{loram}}
\edtext{ipsum}{
  \Aendnote{Another critical note}
  \Bendnote{Other critical note in series B}}
 dolor sit amet, consectetur adipiscing elit. 
 \edtext{Fusce sed dolor libero. Aenean rutrum vestibulum lacus ut pretium. Fusce et auctor lectus. Ut et commodo quam, quis gravida orci. Nullam at risus elementum, suscipit enim a, pellentesque mi}
 {\lemma{Fusce\ldots mi}
 \Aendnote{Critical note pertaining to a long lemma}
 \Bendnote{omit}}. 
Morbi \edtext{commodo}{\Bendnote{quommodo}}, ligula vel consectetur accumsan, massa metus egestas velit, eu fringilla leo ante in turpis. \edtext{Vivamus}{\Bendnote{Vivit}} ut tellus sollicitudin, facilisis ipsum sit amet, tincidunt odio. Maecenas tincidunt dolor sed ante blandit tincidunt. Etiam vulputate ultricies facilisis.
\pend
\endnumbering

\section{A series}
\doendnotes{A}

\section{B series}
\doendnotes{B}


\end{document}
