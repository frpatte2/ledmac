\documentclass{article}
\usepackage{amsmath}
\usepackage[osf,p]{libertinus}
\usepackage{microtype}
\usepackage[pdfusetitle,hidelinks]{hyperref}
\usepackage[english, main=italian]{babel}
\babeltags{english = english}

\usepackage[nocritical,noend,noeledsec,series={A},nofamiliar]{reledmac}

\usepackage{glossaries}
\glsSetCompositor{-}
\makeglossaries
\newglossaryentry{monistero}{name={monistero},description={monastero}}  
\newglossaryentry{magicien}{name={magicien},description={sorcier}}  

\begin{document}

\begin{english}
\title{Glossaries}
\date{}
\maketitle
\begin{abstract}
This file provides example of using \emph{glossaries} package with \emph{reledmac} package.
We use \verb+\edgls+ to add glossary entry referring to page and line number.

Note the use of \verb+\glsSetCompositor+ in the preamble.
\end{abstract}

\end{english}

\beginnumbering
\pstart
In queste nostre contrade fu, ed è ancora, un \edgls{monistero} di donne assai famoso di santità 
(il quale io non nomerò per non diminuire in parte alcuna la fama sua), nel quale, non ha gran 
tempo, non essendovi allora più che otto donne con una badessa, e tutte giovani, era un buono 
omicciuolo d'un loro bellissimo giardino ortolano, il quale, non contentandosi del salario, fatta la 
ragion sua col castaldo delle donne, a Lamporecchio, là ond'egli era, se ne tornò. Quivi, tra gli 
altri che lietamente il raccolsono, fu un giovane lavoratore forte e robusto e, secondo uom di villa, 
con bella persona e con viso assai piacevole, il cui nome era Masetto; e 
domandollo dove tanto tempo stato fosse. Il buono uomo, che Nuto avea nome, 
gliele disse. Il quale Masetto domandò, di che egli il monistero servisse. \edgls{magicien}
\pend
\endnumbering
\printglossaries

\end{document}
