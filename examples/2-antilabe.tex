\documentclass{article}
\usepackage[osf,p]{libertinus}
\usepackage[english]{babel}
\babeltags{english = english}

\usepackage[characterstyle=imprimerie-prose]{thalie}

\usepackage[noeledsec,noend,noledgroup,nofamiliar,nocritical,antilabe]{reledmac}
\setstanzaindents{1,0,1}
\setcounter{stanzaindentsrepetition}{1}
\setcounter{startstanzaindentsrepetition}{2}

\linenumincrement{1}
\firstlinenum{1}
%%retained from the thalie package, modifies the way a character's name is given in the first line of speech
\renewcommand*{\speakswithoutdirection}[1]{%
  \noindent\textsc{#1}\xspace.%
}

% Space after \antilabe macro
\renewcommand{\afterantilabe}{\hspace{1em}}
\begin{document}


\begin{abstract}
This file provides an example of typesetting verse for a dramatic text with reledmac and thalie packages.

We use the thalie features to manage characters.

We use the following features of reledmac to typeset the verses:
\begin{itemize}
  \item \verb+\antilabe+ to show correctly the second half of a split line of verse (antilabe);
  \item \verb+\afterantilabe+ to add more space after \verb+\antilabe+;
  \item \verb+\setcounter{stanzaindentsrepetition}+ and \\  \verb+\setcounter{startstanzaindentsrepetition}+ to have the first line of a reply/stanza indented, but not the following ones;
  \item \verb+\skipnumbering+ to avoid counting the first half of an antilabe in the line counting.
\end{itemize}
\end{abstract}

ddd
%%part of thalie's management of characters
\begin{dramatis}
  \character[cmd={Buckingham}]{Buckingham}
  \character[cmd={Norfolk}]{Norfolk}
\end{dramatis}

% Normal typesetting
\beginnumbering
\stanza\Buckingham Good morrow and well met. How have ye done&
\skipnumbering Since last we saw in France?\&
\stanza\antilabe\Norfolk I thank your grace,&
Healthful, and ever since a fresh admirer&
\skipnumbering Of what I saw there.\&
\stanza\antilabe\Buckingham An untimely ague&
Stayed me a prisoner in my chamber when\&
\endnumbering

\end{document}
