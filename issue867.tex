\documentclass{book}
\usepackage{blindtext}
\usepackage[english]{babel}
\usepackage[noend,series={A,B}]{reledmac}
\listfiles

\setlength{\stanzaindentbase}{20pt} %Play with it later.
\setstanzaindents{5,1,1}
\setcounter{stanzaindentsrepetition}{2}
\newcommand{\stanzaend}{\&}
\sethangingsymbol{\protect\hfill}

\begin{document}
\tableofcontents

   \section {Arthur Schnitzler an Richard Beer-Hofmann, 29. 9. 1894}  \beginnumbering
            \pstart
           Wien  , 29. 9. 94.\pend
           \pstart
           Lieber Richard, {zwei} (due) Karten hab ich Ihnen nach Pallanza geschrieben – das ist doch mehr als Mau? – Sie sind
               offenbar verloren gegangen.\pend
                        \pstart
           Gestern Eröffnung Josefstadt; mit Dank des Herrn Léon   im Frack, mit gekränkter Miene. Sehr amüsant,
               abgesehn vom 1. Akt. –\pend
           \pstart
           Mein Stück – zwei Akte bis auf
               letzte Feile (exclus.) vollendet. Wohl in acht Tagen fertig, – bühnenfertig in etwa
               4 Wochen, bühnenwirksam – wann? –\pend
           \pstart
           Wie fühlen Sie sich? »Fliesst die Arbeit munter fort?« –\pend
           \pstart
           »Zeit«
               soll besorgt werden. – Bitte schreiben Sie häufiger – die Gemäldegalerie, die so
               hoffnungsvoll begonnen, hat rasch geendet. –\pend
           \pstart
           Herzlich der Ihre{\\[\baselineskip]}  {{Richard} entschuldigen – Arthur.}\pend
           \leftskip=0em{}\pstart
           »Aeh, Kamerad, und was machen Weiber?« (Carricaturen, Floh, Bombe, Wiener
                  Witzblatt).\pend
           \stanza{}Und jene schöne, die vor Zeiten Euch\newverse{}Das Wasser auf den Nachttisch Abends stellte –\newverse{}Mit der Madonna holdem Lächeln – denkt\newverse{}Ihr dieses guten Mädchens manchmal noch, –\newverse{}Das sicher manches gegen die Empfängnis,\newverse{}Doch gegen das Beflecktsein gar nichts hatte –?\stanzaend{}\pstart
           Der Obige, was ich leider nicht auf jenes Mädchen beziehn kann.\pend
           \pstart
             {A.}\pend
           \pstart
           (nach Florenz  a posta ferma)\pend
           \endnumbering

              \section {Arthur Schnitzler an Richard Beer-Hofmann, 29. 9. 1894}  \beginnumbering
            \pstart
           Wien  , 29. 9. 94.\pend
           \pstart
           Lieber Richard, {zwei} (due) Karten hab ich Ihnen nach Pallanza geschrieben – das ist doch mehr als Mau? – Sie sind
               offenbar verloren gegangen.\pend
                        \pstart
           Gestern Eröffnung Josefstadt; mit Dank des Herrn Léon   im Frack, mit gekränkter Miene. Sehr amüsant,
               abgesehn vom 1. Akt. –\pend
           \pstart
           Mein Stück – zwei Akte bis auf
               letzte Feile (exclus.) vollendet. Wohl in acht Tagen fertig, – bühnenfertig in etwa
               4 Wochen, bühnenwirksam – wann? –\pend
           \pstart
           Wie fühlen Sie sich? »Fliesst die Arbeit munter fort?« –\pend
           \pstart
           »Zeit«
               soll besorgt werden. – Bitte schreiben Sie häufiger – die Gemäldegalerie, die so
               hoffnungsvoll begonnen, hat rasch geendet. –\pend
           \pstart
           Herzlich der Ihre{\\[\baselineskip]}  {{Richard} entschuldigen – Arthur.}\pend
           \leftskip=0em{}\pstart
           »Aeh, Kamerad, und was machen Weiber?« (Carricaturen, Floh, Bombe, Wiener
                  Witzblatt).\pend
           \stanza{}Und jene schöne, die vor Zeiten Euch\newverse{}Das Wasser auf den Nachttisch Abends stellte –\newverse{}Mit der Madonna holdem Lächeln – denkt\newverse{}Ihr dieses guten Mädchens manchmal noch, –\newverse{}Das sicher manches gegen die Empfängnis,\newverse{}Doch gegen das Beflecktsein gar nichts hatte –?\stanzaend{}\pstart
           Der Obige, was ich leider nicht auf jenes Mädchen beziehn kann.\pend
           \pstart
             {A.}\pend
           \pstart
           (nach Florenz  a posta ferma)\pend
           \endnumbering
              \section {Arthur Schnitzler an Richard Beer-Hofmann, 29. 9. 1894}  \beginnumbering
            \pstart
           Wien  , 29. 9. 94.\pend
           \pstart
           Lieber Richard, {zwei} (due) Karten hab ich Ihnen nach Pallanza  geschrieben – das ist doch mehr als Mau? – Sie sind
               offenbar verloren gegangen.\pend
           \pstart
           (Wer, – ich? (Leon   und Waldberg  , Blumenthal{Blumenthal, Oskar    {Blumenthal, Oskar} (13. 3. 1852 – 24. 4. 1917)  } und
                   \pend
           \pstart
           Gestern Eröffnung Josefstadt{Theater in der Josefstadt  \textbf{Theater in der Josefstadt}, \emph{1. 1. 17884.0}  }; mit Dank des Herrn Léon   im Frack, mit gekränkter Miene. Sehr amüsant,
               abgesehn vom 1. Akt. –\pend
           \pstart
           Mein Stück – zwei Akte bis auf
               letzte Feile (exclus.) vollendet. Wohl in acht Tagen fertig, – bühnenfertig in etwa
               4 Wochen, bühnenwirksam – wann? –\pend
           \pstart
           Wie fühlen Sie sich? »Fliesst die Arbeit munter fort?« –\pend
           \pstart
           »Zeit«
               soll besorgt werden. – Bitte schreiben Sie häufiger – die Gemäldegalerie, die so
               hoffnungsvoll begonnen, hat rasch geendet. –\pend
           \pstart
           Herzlich der Ihre{\\[\baselineskip]}  {{Richard} entschuldigen – Arthur.}\pend
           \leftskip=0em{}\pstart
           »Aeh, Kamerad, und was machen Weiber?« (Carricaturen, Floh, Bombe, Wiener
                  Witzblatt).\pend
           \stanza{}Und jene schöne, die vor Zeiten Euch\newverse{}Das Wasser auf den Nachttisch Abends stellte –\newverse{}Mit der Madonna holdem Lächeln – denkt\newverse{}Ihr dieses guten Mädchens manchmal noch, –\newverse{}Das sicher manches gegen die Empfängnis,\newverse{}Doch gegen das Beflecktsein gar nichts hatte –?\stanzaend{}\pstart
           Der Obige, was ich leider nicht auf jenes Mädchen beziehn kann.\pend
           \pstart
             {A.}\pend
           \pstart
           (nach Florenz  a posta ferma)\pend
           \endnumbering
              \section {Arthur Schnitzler an Richard Beer-Hofmann, 29. 9. 1894}  \beginnumbering
            \pstart
           Wien  , 29. 9. 94.\pend
           \pstart
           Lieber Richard, {zwei} (due) Karten hab ich Ihnen nach Pallanza  geschrieben – das ist doch mehr als Mau? – Sie sind
               offenbar verloren gegangen.\pend
           \pstart
           (Wer, – ich? (Leon   und Waldberg  , Blumenthal{Blumenthal, Oskar    {Blumenthal, Oskar} (13. 3. 1852 – 24. 4. 1917)  } und
                   \pend
           \pstart
           Gestern Eröffnung Josefstadt{Theater in der Josefstadt  \textbf{Theater in der Josefstadt}, \emph{1. 1. 17884.0}  }; mit Dank des Herrn Léon   im Frack, mit gekränkter Miene. Sehr amüsant,
               abgesehn vom 1. Akt. –\pend
           \pstart
           Mein Stück – zwei Akte bis auf
               letzte Feile (exclus.) vollendet. Wohl in acht Tagen fertig, – bühnenfertig in etwa
               4 Wochen, bühnenwirksam – wann? –\pend
           \pstart
           Wie fühlen Sie sich? »Fliesst die Arbeit munter fort?« –\pend
           \pstart
           »Zeit«
               soll besorgt werden. – Bitte schreiben Sie häufiger – die Gemäldegalerie, die so
               hoffnungsvoll begonnen, hat rasch geendet. –\pend
           \pstart
           Herzlich der Ihre{\\[\baselineskip]}  {{Richard} entschuldigen – Arthur.}\pend
           \leftskip=0em{}\pstart
           »Aeh, Kamerad, und was machen Weiber?« (Carricaturen, Floh, Bombe, Wiener
                  Witzblatt).\pend
           \stanza{}Und jene schöne, die vor Zeiten Euch\newverse{}Das Wasser auf den Nachttisch Abends stellte –\newverse{}Mit der Madonna holdem Lächeln – denkt\newverse{}Ihr dieses guten Mädchens manchmal noch, –\newverse{}Das sicher manches gegen die Empfängnis,\newverse{}Doch gegen das Beflecktsein gar nichts hatte –?\stanzaend{}\pstart
           Der Obige, was ich leider nicht auf jenes Mädchen beziehn kann.\pend
           \pstart
             {A.}\pend
           \pstart
           (nach Florenz  a posta ferma)\pend
           \endnumbering
              \section {Arthur Schnitzler an Richard Beer-Hofmann, 29. 9. 1894}  \beginnumbering
            \pstart
           Wien  , 29. 9. 94.\pend
           \pstart
           Lieber Richard, {zwei} (due) Karten hab ich Ihnen nach Pallanza  geschrieben – das ist doch mehr als Mau? – Sie sind
               offenbar verloren gegangen.\pend
           \pstart
           (Wer, – ich? (Leon   und Waldberg  , Blumenthal{Blumenthal, Oskar    {Blumenthal, Oskar} (13. 3. 1852 – 24. 4. 1917)  } und
                   \pend
           \pstart
           Gestern Eröffnung Josefstadt{Theater in der Josefstadt  \textbf{Theater in der Josefstadt}, \emph{1. 1. 17884.0}  }; mit Dank des Herrn Léon   im Frack, mit gekränkter Miene. Sehr amüsant,
               abgesehn vom 1. Akt. –\pend
           \pstart
           Mein Stück – zwei Akte bis auf
               letzte Feile (exclus.) vollendet. Wohl in acht Tagen fertig, – bühnenfertig in etwa
               4 Wochen, bühnenwirksam – wann? –\pend
           \pstart
           Wie fühlen Sie sich? »Fliesst die Arbeit munter fort?« –\pend
           \pstart
           »Zeit«
               soll besorgt werden. – Bitte schreiben Sie häufiger – die Gemäldegalerie, die so
               hoffnungsvoll begonnen, hat rasch geendet. –\pend
           \pstart
           Herzlich der Ihre{\\[\baselineskip]}  {{Richard} entschuldigen – Arthur.}\pend
           \leftskip=0em{}\pstart
           »Aeh, Kamerad, und was machen Weiber?« (Carricaturen, Floh, Bombe, Wiener
                  Witzblatt).\pend
           \stanza{}Und jene schöne, die vor Zeiten Euch\newverse{}Das Wasser auf den Nachttisch Abends stellte –\newverse{}Mit der Madonna holdem Lächeln – denkt\newverse{}Ihr dieses guten Mädchens manchmal noch, –\newverse{}Das sicher manches gegen die Empfängnis,\newverse{}Doch gegen das Beflecktsein gar nichts hatte –?\stanzaend{}\pstart
           Der Obige, was ich leider nicht auf jenes Mädchen beziehn kann.\pend
           \pstart
             {A.}\pend
           \pstart
           (nach Florenz  a posta ferma)\pend
           \endnumbering
              \section {Arthur Schnitzler an Richard Beer-Hofmann, 29. 9. 1894}  \beginnumbering
            \pstart
           Wien  , 29. 9. 94.\pend
           \pstart
           Lieber Richard, {zwei} (due) Karten hab ich Ihnen nach Pallanza  geschrieben – das ist doch mehr als Mau? – Sie sind
               offenbar verloren gegangen.\pend
           \pstart
           (Wer, – ich? (Leon   und Waldberg  , Blumenthal{Blumenthal, Oskar    {Blumenthal, Oskar} (13. 3. 1852 – 24. 4. 1917)  } und
                   \pend
           \pstart
           Gestern Eröffnung Josefstadt{Theater in der Josefstadt  \textbf{Theater in der Josefstadt}, \emph{1. 1. 17884.0}  }; mit Dank des Herrn Léon   im Frack, mit gekränkter Miene. Sehr amüsant,
               abgesehn vom 1. Akt. –\pend
           \pstart
           Mein Stück – zwei Akte bis auf
               letzte Feile (exclus.) vollendet. Wohl in acht Tagen fertig, – bühnenfertig in etwa
               4 Wochen, bühnenwirksam – wann? –\pend
           \pstart
           Wie fühlen Sie sich? »Fliesst die Arbeit munter fort?« –\pend
           \pstart
           »Zeit«
               soll besorgt werden. – Bitte schreiben Sie häufiger – die Gemäldegalerie, die so
               hoffnungsvoll begonnen, hat rasch geendet. –\pend
           \pstart
           Herzlich der Ihre{\\[\baselineskip]}  {{Richard} entschuldigen – Arthur.}\pend
           \leftskip=0em{}\pstart
           »Aeh, Kamerad, und was machen Weiber?« (Carricaturen, Floh, Bombe, Wiener
                  Witzblatt).\pend
           \stanza{}Und jene schöne, die vor Zeiten Euch\newverse{}Das Wasser auf den Nachttisch Abends stellte –\newverse{}Mit der Madonna holdem Lächeln – denkt\newverse{}Ihr dieses guten Mädchens manchmal noch, –\newverse{}Das sicher manches gegen die Empfängnis,\newverse{}Doch gegen das Beflecktsein gar nichts hatte –?\stanzaend{}\pstart
           Der Obige, was ich leider nicht auf jenes Mädchen beziehn kann.\pend
           \pstart
             {A.}\pend
           \pstart
           (nach Florenz  a posta ferma)\pend
           \endnumbering
              \section {Arthur Schnitzler an Richard Beer-Hofmann, 29. 9. 1894}  \beginnumbering
            \pstart
           Wien  , 29. 9. 94.\pend
           \pstart
           Lieber Richard, {zwei} (due) Karten hab ich Ihnen nach Pallanza  geschrieben – das ist doch mehr als Mau? – Sie sind
               offenbar verloren gegangen.\pend
           \pstart
           (Wer, – ich? (Leon   und Waldberg  , Blumenthal{Blumenthal, Oskar    {Blumenthal, Oskar} (13. 3. 1852 – 24. 4. 1917)  } und
                   \pend
           \pstart
           Gestern Eröffnung Josefstadt{Theater in der Josefstadt  \textbf{Theater in der Josefstadt}, \emph{1. 1. 17884.0}  }; mit Dank des Herrn Léon   im Frack, mit gekränkter Miene. Sehr amüsant,
               abgesehn vom 1. Akt. –\pend
           \pstart
           Mein Stück – zwei Akte bis auf
               letzte Feile (exclus.) vollendet. Wohl in acht Tagen fertig, – bühnenfertig in etwa
               4 Wochen, bühnenwirksam – wann? –\pend
           \pstart
           Wie fühlen Sie sich? »Fliesst die Arbeit munter fort?« –\pend
           \pstart
           »Zeit«
               soll besorgt werden. – Bitte schreiben Sie häufiger – die Gemäldegalerie, die so
               hoffnungsvoll begonnen, hat rasch geendet. –\pend
           \pstart
           Herzlich der Ihre{\\[\baselineskip]}  {{Richard} entschuldigen – Arthur.}\pend
           \leftskip=0em{}\pstart
           »Aeh, Kamerad, und was machen Weiber?« (Carricaturen, Floh, Bombe, Wiener
                  Witzblatt).\pend
           \stanza{}Und jene schöne, die vor Zeiten Euch\newverse{}Das Wasser auf den Nachttisch Abends stellte –\newverse{}Mit der Madonna holdem Lächeln – denkt\newverse{}Ihr dieses guten Mädchens manchmal noch, –\newverse{}Das sicher manches gegen die Empfängnis,\newverse{}Doch gegen das Beflecktsein gar nichts hatte –?\stanzaend{}\pstart
           Der Obige, was ich leider nicht auf jenes Mädchen beziehn kann.\pend
           \pstart
             {A.}\pend
           \pstart
           (nach Florenz  a posta ferma)\pend
           \endnumbering
              \section {Arthur Schnitzler an Richard Beer-Hofmann, 29. 9. 1894}  \beginnumbering
            \pstart
           Wien  , 29. 9. 94.\pend
           \pstart
           Lieber Richard, {zwei} (due) Karten hab ich Ihnen nach Pallanza  geschrieben – das ist doch mehr als Mau? – Sie sind
               offenbar verloren gegangen.\pend
           \pstart
           (Wer, – ich? (Leon   und Waldberg  , Blumenthal{Blumenthal, Oskar    {Blumenthal, Oskar} (13. 3. 1852 – 24. 4. 1917)  } und
                   \pend
           \pstart
           Gestern Eröffnung Josefstadt{Theater in der Josefstadt  \textbf{Theater in der Josefstadt}, \emph{1. 1. 17884.0}  }; mit Dank des Herrn Léon   im Frack, mit gekränkter Miene. Sehr amüsant,
               abgesehn vom 1. Akt. –\pend
           \pstart
           Mein Stück – zwei Akte bis auf
               letzte Feile (exclus.) vollendet. Wohl in acht Tagen fertig, – bühnenfertig in etwa
               4 Wochen, bühnenwirksam – wann? –\pend
           \pstart
           Wie fühlen Sie sich? »Fliesst die Arbeit munter fort?« –\pend
           \pstart
           »Zeit«
               soll besorgt werden. – Bitte schreiben Sie häufiger – die Gemäldegalerie, die so
               hoffnungsvoll begonnen, hat rasch geendet. –\pend
           \pstart
           Herzlich der Ihre{\\[\baselineskip]}  {{Richard} entschuldigen – Arthur.}\pend
           \leftskip=0em{}\pstart
           »Aeh, Kamerad, und was machen Weiber?« (Carricaturen, Floh, Bombe, Wiener
                  Witzblatt).\pend
           \stanza{}Und jene schöne, die vor Zeiten Euch\newverse{}Das Wasser auf den Nachttisch Abends stellte –\newverse{}Mit der Madonna holdem Lächeln – denkt\newverse{}Ihr dieses guten Mädchens manchmal noch, –\newverse{}Das sicher manches gegen die Empfängnis,\newverse{}Doch gegen das Beflecktsein gar nichts hatte –?\stanzaend{}\pstart
           Der Obige, was ich leider nicht auf jenes Mädchen beziehn kann.\pend
           \pstart
             {A.}\pend
           \pstart
           (nach Florenz  a posta ferma)\pend
           \endnumbering
              \section {Arthur Schnitzler an Richard Beer-Hofmann, 29. 9. 1894}  \beginnumbering
            \pstart
           Wien  , 29. 9. 94.\pend
           \pstart
           Lieber Richard, {zwei} (due) Karten hab ich Ihnen nach Pallanza  geschrieben – das ist doch mehr als Mau? – Sie sind
               offenbar verloren gegangen.\pend
           \pstart
           (Wer, – ich? (Leon   und Waldberg  , Blumenthal{Blumenthal, Oskar    {Blumenthal, Oskar} (13. 3. 1852 – 24. 4. 1917)  } und
                   \pend
           \pstart
           Gestern Eröffnung Josefstadt{Theater in der Josefstadt  \textbf{Theater in der Josefstadt}, \emph{1. 1. 17884.0}  }; mit Dank des Herrn Léon   im Frack, mit gekränkter Miene. Sehr amüsant,
               abgesehn vom 1. Akt. –\pend
           \pstart
           Mein Stück – zwei Akte bis auf
               letzte Feile (exclus.) vollendet. Wohl in acht Tagen fertig, – bühnenfertig in etwa
               4 Wochen, bühnenwirksam – wann? –\pend
           \pstart
           Wie fühlen Sie sich? »Fliesst die Arbeit munter fort?« –\pend
           \pstart
           »Zeit«
               soll besorgt werden. – Bitte schreiben Sie häufiger – die Gemäldegalerie, die so
               hoffnungsvoll begonnen, hat rasch geendet. –\pend
           \pstart
           Herzlich der Ihre{\\[\baselineskip]}  {{Richard} entschuldigen – Arthur.}\pend
           \leftskip=0em{}\pstart
           »Aeh, Kamerad, und was machen Weiber?« (Carricaturen, Floh, Bombe, Wiener
                  Witzblatt).\pend
           \stanza{}Und jene schöne, die vor Zeiten Euch\newverse{}Das Wasser auf den Nachttisch Abends stellte –\newverse{}Mit der Madonna holdem Lächeln – denkt\newverse{}Ihr dieses guten Mädchens manchmal noch, –\newverse{}Das sicher manches gegen die Empfängnis,\newverse{}Doch gegen das Beflecktsein gar nichts hatte –?\stanzaend{}\pstart
           Der Obige, was ich leider nicht auf jenes Mädchen beziehn kann.\pend
           \pstart
             {A.}\pend
           \pstart
           (nach Florenz  a posta ferma)\pend
           \endnumbering
   \section {Arthur Schnitzler an Richard Beer-Hofmann, 29. 9. 1894}  \beginnumbering
            \pstart
           Wien  , 29. 9. 94.\pend
           \pstart
           Lieber Richard, {zwei} (due) Karten hab ich Ihnen nach Pallanza  geschrieben – das ist doch mehr als Mau? – Sie sind
               offenbar verloren gegangen.\pend
           \pstart
           (Wer, – ich? (Leon   und Waldberg  , Blumenthal{Blumenthal, Oskar    {Blumenthal, Oskar} (13. 3. 1852 – 24. 4. 1917)  } und
                   \pend
           \pstart
           Gestern Eröffnung Josefstadt{Theater in der Josefstadt  \textbf{Theater in der Josefstadt}, \emph{1. 1. 17884.0}  }; mit Dank des Herrn Léon   im Frack, mit gekränkter Miene. Sehr amüsant,
               abgesehn vom 1. Akt. –\pend
           \pstart
           Mein Stück – zwei Akte bis auf
               letzte Feile (exclus.) vollendet. Wohl in acht Tagen fertig, – bühnenfertig in etwa
               4 Wochen, bühnenwirksam – wann? –\pend
           \pstart
           Wie fühlen Sie sich? »Fliesst die Arbeit munter fort?« –\pend
           \pstart
           »Zeit«
               soll besorgt werden. – Bitte schreiben Sie häufiger – die Gemäldegalerie, die so
               hoffnungsvoll begonnen, hat rasch geendet. –\pend
           \pstart
           Herzlich der Ihre{\\[\baselineskip]}  {{Richard} entschuldigen – Arthur.}\pend
           \leftskip=0em{}\pstart
           »Aeh, Kamerad, und was machen Weiber?« (Carricaturen, Floh, Bombe, Wiener
                  Witzblatt).\pend
           \stanza{}Und jene schöne, die vor Zeiten Euch\newverse{}Das Wasser auf den Nachttisch Abends stellte –\newverse{}Mit der Madonna holdem Lächeln – denkt\newverse{}Ihr dieses guten Mädchens manchmal noch, –\newverse{}Das sicher manches gegen die Empfängnis,\newverse{}Doch gegen das Beflecktsein gar nichts hatte –?\stanzaend{}\pstart
           Der Obige, was ich leider nicht auf jenes Mädchen beziehn kann.\pend
           \pstart
             {A.}\pend
           \pstart
           (nach Florenz  a posta ferma)\pend
           \endnumbering
\beginnumbering
   \stanza{}Und jene schöne, die vor Zeiten Euch\newverse{}Das Wasser auf den Nachttisch Abends stellte –\newverse{}Mit der Madonna holdem Lächeln – denkt\newverse{}Ihr dieses guten Mädchens manchmal noch, –\newverse{}Das sicher manches gegen die Empfängnis,\newverse{}Doch gegen das Beflecktsein gar nichts hatte –?\stanzaend{}
\endnumbering
\Blindtext

\beginnumbering

           \stanza{}Und jene schöne, die vor Zeiten Euch\newverse{}Das Wasser auf den Nachttisch Abends stellte –\newverse{}Mit der Madonna holdem Lächeln – denkt\newverse{}Ihr dieses guten Mädchens manchmal noch, –\newverse{}Das sicher manches gegen die Empfängnis,\newverse{}Doch gegen das Beflecktsein gar nichts hatte –?\stanzaend{}

           \endnumbering
           \Blindtext
           \end{document}
